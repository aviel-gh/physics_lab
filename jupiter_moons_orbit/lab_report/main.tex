\documentclass[a4paper, 12pt]{article}

\usepackage[margin=2.5cm]{geometry}

\usepackage{graphicx}
\usepackage[export]{adjustbox}

\usepackage{anyfontsize}
\usepackage{titlesec}
\usepackage{setspace}
\usepackage{parskip}

\usepackage[table]{xcolor}
\usepackage{makecell}

\usepackage{listings}

\usepackage{amsmath}

\usepackage{pdfpages}

\usepackage{hyperref}
\hypersetup{
    colorlinks,
    citecolor=black,
    filecolor=black,
    linkcolor=black,
    urlcolor=black
}
% causes warning because hyperref doesn't like hebrew

\usepackage{polyglossia}
\setmainlanguage{hebrew}
\setotherlanguage{english}
\setmonofont{FreeSerif}
\newfontfamily{\englishfont}{David Libre}
\newfontfamily{\englishfontcode}{JetBrainsMono-Regular}
\newfontfamily{\hebrewfont}{David Libre}

% \begin{english}
% this is where english text can be inserted
% \end{english}

% can also use \textenglish{english text goes here}

% Font - David
% Normal Size - 12pt
% Spacing - 1.15
% Section Size - 16pt
% Subsection Size - 14pt

\titleformat*{\section}{\fontsize{16}{19.2}\selectfont\bfseries}
\titleformat*{\subsection}{\fontsize{14}{16.8}\selectfont\bfseries}
\titleformat*{\subsubsection}{\fontsize{12}{14.4}\selectfont\bfseries}

\setlength{\parindent}{0pt} % removes default paragraph indentation
\setcounter{secnumdepth}{0} % doesn't number sections
\setcounter{tocdepth}{2} % depth of ToC (will only show parts, chapters, sections, subsections)

% ----- Report ----- %

\begin{document}

    {\makeatletter
        \def\@@underline#1{#1}
        \tableofcontents
    \makeatother} % doesn't underline sections in the ToC

    % --- End of Page --- %
    \pagebreak 
    % --- Start of Page --- %

    \section{\underline{מטרת הניסוי}}
    \begin{spacing}{1.15}
        \begin{flushright}

        \end{flushright}
    \end{spacing}


    \begin{figure}[h!]
        
    \end{figure}

    % --- End of Page --- %
    \pagebreak 
    % --- Start of Page --- %

    \section{\underline{רקע תיאורטי}}
    \begin{spacing}{1.15}
        \begin{flushright}
            משוואת כוחות עבור ירח שמסתו $m$ שנע במסלול מעגלי סביב גוף מסיבי שמסתו $M$ (במקרה זה הוא צדק):
        \end{flushright}
    \end{spacing}

    \begin{english}
        $$ \dfrac{GmM}{R^2} = m\dfrac{4\pi^2}{T^2}R $$

        \begin{equation}
            \label{firsteq}
            \dfrac{GM}{4\pi^2} = \dfrac{R^3}{T^2}
        \end{equation}
    \end{english}

    \begin{spacing}{1.15}
        \begin{flushright}
            משוואת כוחות עבור כדור הארץ שמקיף את השמש שמסתה $M_S$ במסלול שרידוסו
            $R_E$ בזמן הקפה $T_E$:
        \end{flushright}
    \end{spacing}

    \begin{english}
        \begin{equation}
            \label{secondeq}
            \dfrac{GM_S}{4\pi^2} = \dfrac{R_E^3}{T_E^2}
        \end{equation}
    \end{english}

    \begin{spacing}{1.15}
        \begin{flushright}
            נחלק משוואה \ref{firsteq} ב- \ref{secondeq}:
        \end{flushright}
    \end{spacing}
    
    \begin{english}
        \begin{flalign*}
            \dfrac{\dfrac{GM}{4\pi^2}}{\dfrac{GM_S}{4\pi^2}} 
            = \dfrac{\dfrac{R^3}{T^2}}{\dfrac{R_E^3}{T_E^2}} 
            \quad\Rightarrow\quad
            \dfrac{M}{M_S} &= \dfrac{R^3}{T^2} \times \dfrac{T_E^2}{R_E^3} \\[0.5em]
            \dfrac{M}{M_S} &= \dfrac{R^3}{R_E^3} \times \dfrac{T_E^2}{T^2} \\[0.5em]
            \dfrac{M}{M_S} &= \dfrac{\dfrac{R^3}{R_E^3}}{\dfrac{T^2}{T_E^2}} &&
        \end{flalign*}

        \begin{equation}
            \label{thirdeq}
            \dfrac{M}{M_S} = \dfrac{\biggl(\dfrac{R}{R_E}\biggl)^3}{\biggl(\dfrac{T}{T_E}\biggl)^2} 
        \end{equation}
    \end{english}

    \begin{spacing}{1.15}
        \begin{flushright}
            קיבלנו משוואה שמתארת את הקשר בין מסת גוף מסיבי (ביחידות מסת שמש)
            לבין היחס של רדיוס מסלולו (ביחידות אסטרונומיות \textenglish{AU}( לזמן המחזור של הקפתו (בשנים).
        \end{flushright}
    \end{spacing}

    % --- End of Page --- %
    \pagebreak 
    % --- Start of Page --- %

    \section{\underline{מערכת הניסוי}}
    \begin{spacing}{1.15}
        \begin{flushright}

        \end{flushright}
    \end{spacing}

    % --- End of Page --- %
    \pagebreak 
    % --- Start of Page --- %

    \section{\underline{מהלך הניסוי}}
    \begin{spacing}{1.15}
        \begin{flushright}

        \end{flushright}
    \end{spacing}

    % --- End of Page --- %
    \pagebreak 
    % --- Start of Page --- %

    \section{\underline{תוצאות הניסוי}}
    \begin{spacing}{1.15}
        \begin{flushright}

        \end{flushright}
    \end{spacing}

    % --- End of Page --- %
    \pagebreak 
    % --- Start of Page --- %

    \section{\underline{עיבוד וניתוח ותוצאות}}
    \begin{spacing}{1.15}
        \begin{flushright}

        \end{flushright}
    \end{spacing}

\end{document}